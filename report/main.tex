% !TEX TS-program = pdflatex
% !TEX encoding = UTF-8 Unicode

% This is a simple template for a LaTeX document using the "article" class.
% See "book", "report", "letter" for other types of document.

\documentclass[11pt]{article} % use larger type; default would be 10pt

\usepackage[utf8]{inputenc} % set input encoding (not needed with XeLaTeX)

%%% Examples of Article customizations
% These packages are optional, depending whether you want the features they provide.
% See the LaTeX Companion or other references for full information.

%%% PAGE DIMENSIONS
\usepackage{geometry} % to change the page dimensions
\geometry{a4paper} % or letterpaper (US) or a5paper or....
% \geometry{margin=2in} % for example, change the margins to 2 inches all round
% \geometry{landscape} % set up the page for landscape
%   read geometry.pdf for detailed page layout information

\usepackage{graphicx} % support the \includegraphics command and options
\graphicspath{ {./images/} }

% \usepackage[parfill]{parskip} % Activate to begin paragraphs with an empty line rather than an indent

%%% PACKAGES
\usepackage{booktabs} % for much better looking tables
\usepackage{array} % for better arrays (eg matrices) in maths
\usepackage{paralist} % very flexible & customisable lists (eg. enumerate/itemize, etc.)
\usepackage{verbatim} % adds environment for commenting out blocks of text & for better verbatim
\usepackage{subfig} % make it possible to include more than one captioned figure/table in a single float
% These packages are all incorporated in the memoir class to one degree or another...

%%% HEADERS & FOOTERS
\usepackage{fancyhdr} % This should be set AFTER setting up the page geometry
\pagestyle{fancy} % options: empty , plain , fancy
\renewcommand{\headrulewidth}{0pt} % customise the layout...
\lhead{}\chead{}\rhead{}
\lfoot{}\cfoot{\thepage}\rfoot{}

%%% SECTION TITLE APPEARANCE
\usepackage{sectsty}
\allsectionsfont{\sffamily\mdseries\upshape} % (See the fntguide.pdf for font help)
% (This matches ConTeXt defaults)

%%% ToC (table of contents) APPEARANCE
\usepackage[nottoc,notlof,notlot]{tocbibind} % Put the bibliography in the ToC
\usepackage[titles,subfigure]{tocloft} % Alter the style of the Table of Contents
\renewcommand{\cftsecfont}{\rmfamily\mdseries\upshape}
\renewcommand{\cftsecpagefont}{\rmfamily\mdseries\upshape} % No bold!

%%% END Article customizations

%%% The "real" document content comes below...

\title{Parallell Particle Simulation}
\author{Dan Ljungström & Daniel Nordin}
\date{VT2022}

\begin{document}

\begin{abstract}
This is a programming project within the course ID1217 at KTH Sweden. The task is to parallelize a toy particle simulator (similar particle simulators are used in mechanics, biology, astronomy.
\end{abstract}


\newpage
\section{Introduction}
The task description is to evaluate four particle simulation programs of O(n2) time complexity, and to improve performance of the programs, i.e. develop and to evaluate the following four particle simulation programs
\\
\\
The four programs consist of:\\
A sequential program that runs in time T = O(n), where n is the number of particles.\\
A parallel program using Pthreads that runs in time close to T/p when using p processors.\\
A parallel program using OpenMP (with/without tasks) that runs in time close to T/p when using p processors.\\
A parallel program using MPI that runs in time close to T/p when using p processors.\\

\section{Improvements}
Here we will declare our evaluations of each of the programs, as well what changes were made to improve the performance of the program.
\subsection{Sequential}
We started out by evaluating the sequential program. This had a time complexity of O(n2) at the start. We had to improve this to O(n).
The time complexity was O(n2) because the program was checking all particles within the simulation at all times. Since the particles are only affected by the particles in a close proximity, we have to reduce the amount of checks being made.
The easiest way to do this, is to implement a way to see where each particle is positioned. We chose to create a grid based on the size and amount of particles. Each particle was then assigned a specific grid, and we could now check what particles are within that grid.



\end{document}